%%%%%%%%%%%%%%%%%%%%%%%%%%%%%%%%%%%%%%%%%%%%%%%%%%%%%%%%%%%%%%%%%%%%%%%%%
%%
%W  intro.tex             GAP documentation                  Bettina Eick
%W                                                          Werner Nickel
%%
%H  $Id: intro.tex,v 1.9 2003/10/24 12:05:36 werner Exp $
%%

%%%%%%%%%%%%%%%%%%%%%%%%%%%%%%%%%%%%%%%%%%%%%%%%%%%%%%%%%%%%%%%%%%%%%%%%%%%
\Chapter{Introduction to polycyclic presentations}

Let $G$ be a polycyclic group and let $G = C_1 \rhd C_2 \ldots C_n\rhd
C_{n+1} = 1$ be a *polycyclic series*, that  is, a subnormal series of
$G$ with non-trivial cyclic factors.  For $1  \leq i \leq n$ we choose
$g_i \in C_i$ such  that $C_i =  \langle  g_i, C_{i+1}  \rangle$.  Then
the
sequence $(g_1,   \ldots, g_n)$  is   called a  *polycyclic generating
sequence of  $G$*.  Let $I$ be  the set of those  $i  \in \{1, \ldots,
n\}$ with $r_i := [C_i : C_{i+1}]$ finite.  Each element of $G$ can be
written <uniquely> as  $g_1^{e_1}\cdots g_n^{e_n}$ with $e_i\in\Z$ for
$1\leq i\leq n$ and $0\leq e_i\<r_i$ for $i\in I$.

Each polycyclic    generating sequence   of  $G$   gives  rise  to a
*power-conjugate  (pc-)   presentation*  for  $G$  with  the  conjugate
relations
$$g_j^{g_i} = g_{i+1}^{e(i,j,i+1)} \cdots g_n^{e(i,j,n)}
                    \hbox{ for } 1 \leq i \< j \leq n,$$
$$g_j^{g_i^{-1}} = g_{i+1}^{f(i,j,i+1)} \cdots g_n^{f(i,j,n)}
                    \hbox{ for } 1 \leq i \< j \leq n,$$
and the power relations
$$g_i^{r_i} = g_{i+1}^{l(i,i+1)} \cdots g_n^{l(i,n)}
                    \hbox{ for } i \in I\.$$
\bigskip

Vice versa, we say that a group $G$ is defined by a pc-presentation if
$G$  is  given by  a  presentation of  the  form  above on  generators
$g_1,\ldots,g_n$.  These  generators are the  *defining generators* of
$G$.  Here, $I$  is the set of  $1\leq i\leq n$ such that  $g_i$ has a
power relation.  The positive integer $r_i$ for $i\in I$ is called the
*relative order* of $g_i$. If  $G$ is given by a pc-presentation, then
$G$  is polycyclic.  The subgroups  $C_i  = \langle  g_i, \ldots,  g_n
\rangle$ form a  subnormal series $G = C_1 \geq  \ldots \geq C_{n+1} =
1$  with cyclic  factors  and  we have  that  $g_i^{r_i}\in C_{i+1}$.  
However, some of the factors of  this series may be smaller than $r_i$
for $i\in I$ or finite if $i\not\in I.$

If $G$ is defined by a pc-presentation, then
each  element of  $G$    can be described   by   a word of   the  form
$g_1^{e_1}\cdots g_n^{e_n}$ in the defining generators with $e_i\in\Z$
for $1\leq i\leq n$ and $0\leq e_i\<r_i$ for $i\in I$.  Such a word is
said to be in *collected  form*.  In general, an  element of the group
can  be  represented by   more  than   one   collected word.    If the
pc-presentation  has  the property    that each element  of  $G$   has
precisely one word in collected form, then  the presentation is called
*confluent* or *consistent*.  If that is the case, the generators with
a power  relation correspond precisely to  the  finite factors  in the
polycyclic series and $r_i$ is the order of $C_i/C_{i+1}$.

The GAP 4 package {\sf polycyclic} is designed for computations with
polycyclic groups which are given by consistent pc-presentations. In
particular, all the functions described below assume that we compute
with a group defined by a consistent pc-presentation. See Section 
"Collectors" for a routine that checks the consistency of a
pc-presentation. 

A pc-presentation can be interpreted as a *rewriting system* in the
following way.  One needs need to add a new generator $G_i$ for each
generator $g_i$ together with the relations $g_iG_i = 1$ and $G_ig_i =
1$.  Any occurrence in a relation of an inverse generator $g_i^{-1}$
is replaced by $G_i$.  In this way one obtains a monoid presentation
for the group $G$.  With respect to a particular ordering on the set
of monoid words in the generators $g_1,\ldots g_n,G_1,\ldots G_n$, the
*wreath product ordering*, this monoid presentation is a rewriting
system.  If the pc-presentation is consistent, the 
rewriting system is confluent.

In this package we do not address this aspect of pc-presentations
because it is of little relevance for the algorithms implemented here.
For the definition of rewriting systems and confluence in this context
as well as further details on the connections between pc-presentations
and rewriting systems we recommend the book \cite{Sims94}.


